%%%%%%%%%%%%%%%%%%%%%%%%%%%%%%%%%%%%%%%%%%%%%%%%%%%%%%%%%%%%%%%%%%%%
% Technologies and Implementation
%%%%%%%%%%%%%%%%%%%%%%%%%%%%%%%%%%%%%%%%%%%%%%%%%%%%%%%%%%%%%%%%%%%%

\chapter{Implementation}
  \label{Implementation}
The following chapter explains how the user interface is structured and which options it holds for the user, after describing the intention of this project.
\section{Motivation and intent}
As mentioned in the introduction, figuring out a linear layout with certain properties by hand is a time consuming, error-prone task. This is why efforts have been made to automate this process. The discovery that SAT formulas could do this quite efficiently \cite{Bekos2015TheBE} was a big achievement.\\
Up to now the graphs had to be transformed into a textual representation in order to then be translated tp a SAT instance and passed to the solver. The intent of this project was therefore to develop an editor where a graph can be created as a drawing, then be passed to a translating routine and a solver. After receiving the solver result, the embedding should be displayed again in a visually appealing way.\\
Furthermore it was an ambition to check a graph for specific linear layouts, since randomized SAT-sovers produce random linear layouts. Thankfully this can be easily achieved by expanding the SAT instance by clauses that lead the SAT-solver in the desired direction. Therefore the editor needed an option to impose such constraints upon the future linear layout.\\
It was decided that the editor should be webbased in order achieve broad availability, since then nothing more than a fairly modern browser is needed.\\
\section{Graph Editor}
\begin{figure}[!h]
\begin{center}
\includegraphics[width=1\textwidth]{figures/figIndex/overviewIndex.jpg}
\caption{Overview over the user interface}
\label{img:overviewIndex}
\end{center}
\end{figure}
\subsection{Overwiew}
The user interface of this tool consists of three main areas: the toolbar on top, the interactive graph editor in the middle and the configuration panel, where the user can specify the linear layout he or she wishes to compute.\\
\subsection{Graph Editor}
The graph editor is the center area of the user interface. It displays a grid by default which can be disabled.\\
\subsubsection{User Interaction}
Nodes are created by clicking on the canvas. Edges can only be created between two existing nodes, by dragging a line from the source node to the target. By default, double edges are forbidden but can be allowed by the user through the \textbf{tools} submenu.\\
On creation, each node and edge is assigned a label, counting from 0. These labels can be changed by the user.\\
Bends in the edges can be achieved by releasing the mouse button wherever the bend should be. In big graphs this enables a tidier drawing.\\
If the grid is visible the graph elements encourage certain positions on the grid by clipping to the dots of the grid when moved, but still every element can be placed freely on the canvas.\\
The elements of a graph can be selected and then copied, cut, pasted and deleted. Edges can only be copied when duplicate edges are allowed or when at least one of the corresponding nodes is selected, too.\\
\subsubsection{Identifier of elements}
To compute the linear layout of the graph it is necessary to have a unique id for each element of the graph, so the solver can distinguish the different nodes and edges.\\
On creation every node and edge is assigned a label, which is displayed in the graph editor and can be changed by the user without restriction. This means that labels are not unique and therefore can not be used to identify elements.\\
Invisible for the user each node and edge also is assigned a tag, a feature provided by the \textit{yFiles for HTMl} library. These tags can not be changed by the user and are designed to be unique. Nodes get the next free number as a tag, edges get the tag "a-(0)-b", where a is the tag of the source node and b is the tag of the target node. The \textit{(0)} in the middle of each edge tag is necessary if the user chooses to allow duplicate edges. A duplicate edge would then be assigned the tag "a-(1)-b", ensuring the uniqueness.
\subsection{Toolbar}
On the left section of the toolbar the user can choose from submenus, the options being a \textit{file}, \textit{edit}, \textit{view}, \textit{layout} as well as a link to the about information of this page and a button to toggle the visibility of the configuration panel.
\subsubsection{The file submenu}
\begin{tabular}{p{0.05\textwidth}p{0.95\textwidth}}
\includegraphics[scale=0.6]{figures/icons/new.png} & Clears the canvas and configuration panel to create an entirely new graph\\
\includegraphics[scale=0.6]{figures/icons/download.png}& Save the current graph into the users local file system\\
\includegraphics[scale=0.6]{figures/icons/upload.png}& Load a graph from the users local file system\\
\includegraphics[scale=0.6]{figures/icons/export.png}& Export the graph to either a .png- or .pdf-file\\
\includegraphics[scale=0.6]{figures/icons/server.png} &Change the server to which the graph should be passed on computation. This is a feature for people who use this tool frequently and prefer to install a local copy of the server. The current server setting is displayed in the top right corner, next to the stats button. 
\end{tabular}
\subsubsection{The edit submenu}
\begin{tabular}{p{0.05\textwidth}p{0.95\textwidth}}
 \includegraphics[scale=0.6]{figures/icons/undo.png} & Erases the last changes on the canvas\\
 \includegraphics[scale=0.6]{figures/icons/redo.png} & Redoes the last undone changes on the canvas\\
 \includegraphics[scale=0.6]{figures/icons/copy.png} & Copies the current selection of elements, also achievable by "ctrl + c"\\
 \includegraphics[scale=0.6]{figures/icons/paste.png} & Pastes formerly cut or copied items to the canvas, also achievable by "ctrl + v"\\
 \includegraphics[scale=0.6]{figures/icons/cut.png} & Cuts the current selection of elements, also achievable by "ctrl + x" \\
 \includegraphics[scale=0.6]{figures/icons/select_all.png} & 
Selects all elements currently on the canvas, also achievable by "ctrl + a"\\
\includegraphics[scale=0.6]{figures/icons/delete_all.png} & Deletes the currently selected elements, also achievable by "del" key\\[12pt]
\includegraphics[scale=0.6]{figures/icons/marquee_all.png}  \includegraphics[scale=0.6]{figures/icons/marquee_nodes.png}  \includegraphics[scale=0.6]{figures/icons/marquee_edges.png} & This set of buttons changes which elements of the graph should be selected when a rectangle is dragged over the canvas. The first means all elements are selected, the second means only nodes are selected, the last means that only edges are selected. This is convenient when a constraint needs to be imposed on a large group of nodes or edges.
\end{tabular}
\subsubsection{The view submenu}
\begin{tabular}{p{0.05\textwidth}p{0.95\textwidth}}
\includegraphics[scale=0.6]{figures/icons/zoomin.png} &  Zooms into the canvas\\
\includegraphics[scale=0.6]{figures/icons/zoomout.png}&  Zooms out of the canvas\\
\includegraphics[scale=0.6]{figures/icons/focus.png} & Focusses the graph in the center of the canvas\\
\includegraphics[scale=0.6]{figures/icons/grid.png} & Toggles visibility of the grid.
\end{tabular}
\subsubsection{The layout submenu}
The layout submenu holds several algorithms to transform the graph. The most widely used layouts for graphs were used for this, containing \textit{hierarchic}, \textit{organic}, \textit{orthogonal}, \textit{circular}, \textit{tree}, \textit{balloon} and \textit{radial} layouts.
\subsubsection{The tools submenu}
\begin{tabular}{p{0.05\textwidth}p{0.95\textwidth}}
\includegraphics[scale=0.6]{figures/icons/resize_nodes.png}& Toggles, whether nodes are resizable or not\\
\includegraphics[scale=0.6]{figures/icons/doubleEdges.png}& Toggles, whether duplicate edges are allowed or not\\
\includegraphics[scale=0.6]{figures/icons/stellation.png}& When no nodes are selected, this stellates every face of the graph. That means a new node is placed in the center each face of the graph and edges connect the new node to each node of the face. When one or more nodes are selected, the new node is connected to each of the selected node (see \autoref{img:stell})\\
\includegraphics[scale=0.6]{figures/icons/three_stellation.png}& When less than three nodes are selected, three nodes are inserted into every face of the graph that is bounded by three edges. The new nodes are connected to themselves and the three nodes of the face, as to be seen in \autoref{img:stell}\\
\includegraphics[scale=0.6]{figures/icons/edgeStellation.png}& If no edge is selected, every above every edge a new node is created, which is connected to the two end nodes of the edge. If a set of edges is selected, this only applies to the selection\\
\end{tabular}
\begin{figure}
\begin{center}
\includegraphics[width=\textwidth]{figures/figIndex/stellation.png}
\caption{Stellation of a triangular face}
\label{img:stell}
\end{center}
\end{figure}
\subsubsection{Stats}
% stats panel 
The stats panel gives information of the current graph. First of all it states how many vertices and edges the graph contains. Furthermore it provides information, whether a graph is \textit{planar}, \textit{connected}, \textit{acyclic}, a  \textit{tree} and a \textit{forest}. The algorithms for these properties are provided by \textit{yFiles for html}.
\begin{figure}[!h]
\begin{center}
\includegraphics[width=1\textwidth]{figures/figIndex/StatsPanels.jpg}
\caption{Stats of a graph}
\label{img:stats}
\end{center}
\end{figure}
\subsubsection{Compute}
After clicking on the \textit{compute} button, a dialog opens to ask the user whether she would like to proceed to computing the linear layout as it is defined right now. The user then can choose to first save the graph, proceed or cancel.\\
If the computation is wished, the graph including its constraints and page settings are sent to the server with an ajax-request. Since the request is processed asynchronously by the server, it returns the id of the future embedding right away. This response triggers a redirection to the second page with the newly acquired id as a hash parameter.\\
If the server returns an error, the error message is displayed in a dialog and the user is not redirected.
% ajax request, transformation of data, 
\subsection{The configuration panel}
\begin{figure}[!h]
\begin{center}
\includegraphics[width=1\textwidth]{figures/figIndex/ConfigPanel.jpg}
\caption{Configuration panel}
\label{img:confPan}
\end{center}
\end{figure}
\subsubsection{Page properties}
\label{imp_pages}
% different page constraints etc
In this panel the user can specify on how many pages the graph should be attempted to be embedded, by checking or unchecking the corresponding checkboxes.\\
The type and layout of each (checked) page can be chosen by two select menues. For types, the options \textit{stack} and \textit{queue} are available, for layouts \textit{maximal} (meaning unrestricted), \textit{tree}, \textit{forest} and \textit{matching} (meaning a dispersable embedding) are possible.
\subsubsection{Constraints}
\label{imp_constr}
The key feature of this project is the possibility to lso impose constraints on the linear layout. This is needed because the SAT-Solver produces solutions to the formula on random and sometimes a researcher might want to check whether a layout with specific properties exists. For an example see the \autoref{Exp}.\\
The constraints are based upon a constraint class which yields a subclass for each available constraint.\\
Constraints are created via the context menu that opens whenever a selection of nodes or edges is right clicked, see the context menu in \autoref{img:constraints}.
During the runtime all active constraints are saved in an array and displayed as so called tags in the \textit{Tag-It}\footnote{\url{http://aehlke.github.io/tag-it/}} plugin, which is configured so the tags can be deleted but not edited.\\
If the user choses to delete an element from the graph the constraints corresponding to this element are also deleted, after reminding the user of this and giving the opportunity to cancel the deletion.
\begin{figure}
\begin{center}
\includegraphics[width=\textwidth]{figures/figIndex/Constraints.jpg}
\caption{The available constraints in context menues}
\label{img:constraints}
\end{center}
\end{figure}
\subsubsection{Restrict the linear order}
To restrict the linear order of the vertices, the user can impose the following constraints:
\begin{enumerate}
\item \textbf{Predecessor}
With this constraint the user can specify a relative order of the nodes by defining one node as the predecessor of another. It implicitly also provides successorship, by reversing the predecessor relation.
\item \textbf{Consecutivity} With this constraint two nodes can be required to be consecutive in the linear layout. It does not restrain the order of the nodes further.
\item \textbf{Partial order} When a sequence of at least two nodes is selected, an absolute partial order of these nodes can be either required or forbidden, meaning the exact order in which the nodes have been selected, as to be seen in \autoref{img:partOrder}. If the nodes are not selected individually the order is determined by the time of creation. For easier usage the order in question is displayed in a dialog. 
\begin{figure}
\begin{center}
\includegraphics[width=\textwidth]{figures/figIndex/PartialOrder.jpg}
\caption{Partial order dialog}
\label{img:partOrder}
\end{center}
\end{figure}
\end{enumerate}
\newpage
\subsubsection{Restrict the placement of edges}
\begin{enumerate}
\item \textbf{Assign edges to certain pages} 
\begin{figure}
\begin{center}
\includegraphics[width=\textwidth]{figures/figIndex/Assign.jpg}
\caption{Edge assignment dialog}
\end{center}
\end{figure}
When at one or more edges are selected, right clicking and selecting "Assign to.." opens a dialog, where the user can choose on which pages of the embedding these edges can be located. The set of selected edges is not necessarily on the same page, if more than one page is selected.  
\item \textbf{Assign edges to the same page} Similar to assigning to certain pages, the user can also specify that the selected edges should be placed on the same page. 
\item \textbf{Assign edges to different pages}
As long as the user does not select more edges then there are pages available, the user can assure that all selected edges are on different half planes of the layout by using this constraint.
\item \textbf{Assign the edges incident to certain nodes}
\begin{figure}
\includegraphics[width=\textwidth]{figures/figIndex/AssignAdj.jpg}
\caption{Incident edges dialog}
\end{figure}
When the user rightclickes on two or more nodes, she can also constraint all edges incident to these nodes.\\
If exactly two vertices, say \textit{u} and \textit{v}, are selected, the user can choose to restrain all edges incident to these nodes or to restrain the edges that will end either in the interval between \textit{u} and \textit{v} or the interval between \textit{v} and \textit{u}. If more than two edges are selected these two latter options are not available.\\
\end{enumerate}
It is important to note that the corresponding constraints do only show up if the selection is exclusively nodes, respectively edges. Otherwise the context menu would be too unwieldy to use efficiently.
\newpage 
\section{Linear Layouts Viewer}
\subsection{Overview}
The viewer is structured similarly to the editor, with a toolbar on top, the viewing area in the middle and a panel on the bottom, where the imposed constraints are displayed and the appearance of the graph can be modified for easier examination.
\begin{figure}[!h]
\begin{center}
\includegraphics[width=1\textwidth]{figures/figSecond/OverviewSecond.jpg}
\caption{Overview of the viewing page}
\label{img:plzhltr}
\end{center}
\end{figure}
% how to transform graph, difficulty with arc edges, reading in constraints and pages etc, hash systems
\subsection{Linear layouts viewer}
\label{viewerOV}
% non editable but clickable
As the user accesses the second page, either directly or by redirection from the editor the url is checked for the hash parameter, where the id of the desired embedding should be located. If there is no id specified an error dialog shows up, otherwise an ajax-request is sent to the server.\\
The response is a JSON-encoded object and contains the following fields \cite{linearLayoutApi}\\[12px]
\begin{tabular}{l p{0.7\textwidth}}
id & the id of the embedding\\
graph & the 64-byte encoded graph that was sent to the server\\
pages & the pages of the embedding as JSON objects in a list, each with the attributes \textit{id}, \textit{type} and \textit{layout}\\
constraints & a list of constraints as JSON objects. Each constraint object has the attributes "type", "modifier" and "arguments".\\
status & a string, either "IN\_PROGRESS" or "FINISHED", determining if the server finished the computation of the embedding\\
vertex order & a list of strings, where each string is the tag of a vertex. This list is ordered like the vertices need to be ordered along the spine of the linear layout\\
satisfiable & a boolean, determining whether the graph was embeddable as proposed\\
rawSolverResult & the string as returned by the SAT solver\\
created & a timestamp in string form, e.g. \textit{'2019-06-25T09:50:13.122499+00:00'}
\end{tabular}\\[12pt]
After the website received the response from the server it checks the status of the response. Should it be "IN\_PROGRESS", another request is sent after 5 seconds until the server finished computation. Then as the second step the website checks if the graph was embeddable in the way it was defined, which is stated in the field \textit{satisfiable}. If it is not embeddable a dialog shows up, allowing the user to go back to editing the graph.\\
\begin{figure}
\begin{center}
\includegraphics[width=\textwidth]{figures/figSecond/NotEmbeddable.jpg}
\caption{Dialog when a graph is not embeddable as desired}
\end{center}
\label{notSat}
\end{figure}
\noindent If it was indeed embeddable the website proceeds to modify the graph to display the linear layout in the following steps:
\begin{enumerate}
\item The graph is decoded and loaded into the viewer
\item The vertices are ordered according to the "vertex order" field of the responded embedding
\item The edges are divided into arrays representing each half plane of the graph, according to the "assignment" field\\
\item Each edge is assigned the "Arc Edge Style" provided by \textit{yFiles for HTML}. In order to display the edges correctly one extra step is needed: the position of each arc is determined by the source and target nodes, that is to say edges with a source node right to the target node are displayed beyond the nodes and when the source node is left to the target the edge is displayed above the nodes. As the representation of a linear layout requires all edges of one page to be either above or below the spine, the source and targed nodes of the affected edges have to be switched.
\item For each array representing a page of the embedding, all edges in this array get assigned a color. Each page is positioned aternatingly above and below the spine by changing the height attribute of the affected arcs.\\
\end{enumerate}
\subsection{Toolbar}
% mostly the same as before but less
The toolbar of the second page is a thinned out version of the editors toolbar.\\
The \textbf{file} submenu does no longer allow to load in graphMLs but still provides the saving and exporting options.\\[12pt]
The \textbf{view} tab still has the same options as before, so the user can zoom in and out, center the graph on the screen and toggeling the visibility of the grid.\\
In addition to this it also provides the user with the option to examine the node neighborhood and edge adjacencies in the graph, as explained in \autoref{NandA}.\\
The \textit{about} dialog can still be accessed on the second page and similar to the first page the user can hide the constraints panel at the bottom enlarge the graph viewing area. The \textit{stats} button is in the same place as before, too.\\
Instead of a \textit{compute} button, the toolbar contains a \textit{back to edit} button that takes the user back to the editor, to either edit the original layout or the linear layout, which the user can choose in a dialog. This redirection is explained in \autoref{BtoEdit}\\
\subsubsection{Neighborhood and adjacency dialogs}
\label[NandA]
% neighborhood of a node, adjacency of an edge, necessary for analyzing
The \textit{node neighborhood} and the \textit{edge adjacency} buttons in the \textbf{view} tab of the toolbar, each open a dialog. \\
\begin{itemize}
\item The \textbf{node neighborhood} dialog holds information about the node that is currently selected and is only updated as a different, single node is selected. It lists the neighboring nodes and displays them in the color of the edge that connects the selected node to its neighbor.
\item The \textbf{edge adjacency} dialog tells the user what is the source and end node of the edge currently selected. Similar to the \textit{node neighborhood} it is only updated, when another edge is selected.
\end{itemize}
These dialogs proved to be useful to observe patterns in large embeddings, since following single edges turned out to be cumbersome.
\begin{figure}[!h]
\begin{center}
\includegraphics[width=1\textwidth]{figures/figSecond/NeighborAdjac.jpg}
\caption{Neighborhood and Adjacency}
\label{img:plzhltr}
\end{center}
\end{figure}

\subsubsection{Returning to edit the graph}
\label{BtoEdit}
\begin{figure}[!h]
\begin{center}
\includegraphics[width=1\textwidth]{figures/figSecond/backtoEditing.jpg}
\caption{Return to edit dialog}
\label{img:editDialog}
\end{center}
\end{figure}
When the the "Return to edit" button is clicked a dialog as seen in \autoref{img:editDialog} opens. The user can choose to either edit the linear layout or the original embedding of the graph. After choosing, a redirection back to the editing page is triggered.
The correct redirection is obtained again by adding hash parameters to the url. If the user choses to edit the original layout, the url gets extended by "\#or" and the id of the graph otherwise it is "\#ll" followed by the id. That way a user can also access both layouts later on and even save them as a bookmark.\\
The editing page, recognizing if there is a hash parameter set, sends an ajax-request to the server similar to the viewing page. If the user wishes to edit the linear layout the same processes as described in \autoref{viewerOV} are triggered. When the original embedding is to be edited, the graph is loaded into the editor. The edges still get colored in the respective page color, in order to make examination easier, otherwise the graph remains unchanged.\\
% displaying graph, hash system
\subsection{The configuration panel}
\subsubsection{Constraints}
In the bottom part of the page, the constraints defined in the creation of the graph are displayed in a similar fashion to the editing page, except that the user can not delete a constraint. The constraints read in from the server response.
\subsubsection{Page representation}
For easier observation of the embedding, the graphs representation can be modified. Each page of the embedding can be hidden by unchecking the corresponding checkbox. Furthermore the position and color of each set of edges can be changed, the first by selecting either \textit{above} or \textit{below} in a select menu and the latter with the \textit{Colorpick} plugin\footnote{https://github.com/philzet/ColorPick.js}.
\begin{figure}[!h]
\begin{center}
\includegraphics[width=1\textwidth]{figures/figSecond/ConfigPanel.jpg}
\caption{Pages and Constraints Panel}
\label{img:plzhltr}
\end{center}
\end{figure}

\clearpage
