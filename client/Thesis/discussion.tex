%%%%%%%%%%%%%%%%%%%%%%%%%%%%%%%%%%%%%%%%%%%%%%%%%%%%%%%%%%%%%%%%%%%%
% Diskussion und Ausblick
%%%%%%%%%%%%%%%%%%%%%%%%%%%%%%%%%%%%%%%%%%%%%%%%%%%%%%%%%%%%%%%%%%%%

\chapter{Discussion and Outlook}
  \label{Discussion}

All in all, the project fulfilled the expectations. The objective was a graph editor that is easy to use and allows the user to impose constraints on the linear layout. The resulting framework is intuitively operable, well structured and performs satisfactorily. Furthermore it proved its worth in the attempt to find a planar graph that is embeddable in four stacks, as described in \autoref{Exp}, even though the graph in question was not found.\\
Nevertheless, through using the framework regularly and for large graphs, it turned out to be inefficient in some aspects.\\
One of the deficiencies was that stellating faces of a graph (a process that is used regularly in the research of linear layouts) turned out to be very cumbersome, considering one had to add a node to each face and then connect this new node to each node delimiting the face. This insight lead to the implementation of the stellation buttons within the tools submenu (see \autoref{toolsSub}).\\
The other big issue concernes the creation of constraints. Big graphs, like those described in \autoref{Exp}, usually come with a considerable amount of constraints. Creating those constraints by hand is cumbersome and time consuming and demands a more efficient solution.\\
One considered option was to add the constraints to the copy and paste commands, so when a set of elements is pasted all corresponding constraints should be copied as well. While this is certainly possible, it might turn out to be rather hindering than helpful. The other idea on this behalf was to allow text input to the constraints field, which is actually supported by the \textit{Tag-It} plug-in used for this feature. This, with added auto completion, would make the creation of constraints faster. \\
In summary, while the framework has all desired features and works as planned, there is still room for improvement in user friendliness. Additionally, as the subject of research will change over the course of time, the framework will change further due to changing demands.\\


