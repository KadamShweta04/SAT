%%%%%%%%%%%%%%%%%%%%%%%%%%%%%%%%%%%%%%%%%%%%%%%%%%%%%%%%%%%%%%%%%%%%
% Diskussion und Ausblick
%%%%%%%%%%%%%%%%%%%%%%%%%%%%%%%%%%%%%%%%%%%%%%%%%%%%%%%%%%%%%%%%%%%%

\chapter{Discussion and Outlook}
  \label{Discussion}

All in all this project is considered a success. We achieved the intention of developing an easily accessible graph editor to check graphs for linear layouts. 
The editor is not overcrowded and can be operated intuitively. Though the constraints require basic knowledge about linear layouts, it is safe to assume that only people with this knowledge or at least the will to acquire the knowledge, will use it.\\
The framework was very useful in the experiments described in \autoref{Exp}. It enabled the creation of the graphs and the observation of the linear layouts so patterns and symmetries could be found.\\ 
Nevertheless, through using the tool regularly and for large graphs, it proved to be inefficient in some aspects.\\
One of the deficiencies was that stellating faces of a graph (a process that is used regularly in the research of linear layouts) turned out to be very cumbersome, considering one had to add a node to each face and then connect this new node to each node delimiting the face. This insight lead to the implementation of the stellation buttons within the tools submenu (see \autoref{xyz}).\\
The other big issue concernes the creation of constraints. On big graphs, for example the graphs that were proposed by Prof. Dr. Yannakakis \cite{yannakakis1986four} and explained in \autoref{Exp}, there were at one point several hundred constraints to be created, which would have been very time consuming if done by hand using the context menu. While this issue could be resolved by writing a script to create these constraints as their XML representation and then copying into the already created .graphML file, this is very error prone and not user friendly.\\
A considered solution was to add the constraints to the copy and paste commands, so when a set of elements is pasted all corresponding constraints should be as well. While this is certainly possible, it might turn out to be rather hindering than helpful. The other idea this behalf was to allow text input to the constraints field, which is actually supported by the \textit{Tag-It} plug-in used for this feature. This would make the creation of constraints faster and more effective. As \textit{Tag-It} even offers the possibility for auto completion, the user wouldn't even have to remember the constraints in their entirety. \\
Furthermore for now, some of the constraints are tailor-made for the attempt to find a graph that does not admit to a 3-stack layout. It is safe to assume, that in the future different problems will need solving and new constraints will be added. Ultimately the options will overcrowd the context menu and new means of creating constraints will be required.
In conclusion, though the framework is easily usable and reasonably efficient, there is room for improvement.


