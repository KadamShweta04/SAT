%%%%%%%%%%%%%%%%%%%%%%%%%%%%%%%%%%%%%%%%%%%%%%%%%%%%%%%%%%%%%%%%%%%%
% Einleitung
%%%%%%%%%%%%%%%%%%%%%%%%%%%%%%%%%%%%%%%%%%%%%%%%%%%%%%%%%%%%%%%%%%%%

\chapter{Introduction}\label{Introduction}
A linear layout, in graph theory, is a way of drawing a graph where its vertices are positioned on a line (called the \textit{spine}) and its edges are arcs residing on half planes that are delimited by the line. A linear layout is accompanied by a subdivision of the edges, where each subset occupies one half plane of the graph.\\
Linear layouts are distinguished by the characteristics of subgraphs on each half plane. 




