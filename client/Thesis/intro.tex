%%%%%%%%%%%%%%%%%%%%%%%%%%%%%%%%%%%%%%%%%%%%%%%%%%%%%%%%%%%%%%%%%%%%
% Einleitung
%%%%%%%%%%%%%%%%%%%%%%%%%%%%%%%%%%%%%%%%%%%%%%%%%%%%%%%%%%%%%%%%%%%%

\chapter{Introduction}\label{Introduction}

In graph theory there are \textcolor{red}{several / a few} methods of drawing and analyzing graphs. \\
\textcolor{red}{One of those is linear layouts}, where all vertices are placed on one line in some order and all edges are drawn as half circles above and below this line.\\
The interesting point of these linear layouts then is to determine an order of the vertices in which the edges connecting these are laid out in a special way. \textcolor{red}{I go more into detail about the different constraints in the following chapter}.\\
As much as the \textcolor{red}{field} of linear layouts is an \textcolor{red}{interesting} topic it is also quite tiresome to find a linear layout with a particular attribute by hand, since the researcher would have to either strategically determine for each edge where it should be placed to obtain the desired attribute or iterate through a good portion of the permutations of the vertices, to find this attribute by \textcolor{red}{chance / accident}.\\
For this exact reason researchers found different ways to automate the computation of these linear layouts \textcolor{green}{REFERENCE}.\\
\textcolor{red}{A few years ago} the \textcolor{red}{Arbeitsbereich / Lehrstuhl für Algorithmen} of University of Tübingen proposed an approach to automation by formulating the desired layout as a SAT-formula \textcolor{green}{REFERENCE} and passing it to a SAT-Solver. While this was a notable achievement \textcolor{red}{to the field} it still needed a lot of manual work to convert a graph into the SAT-Formula

