%%%%%%%%%%%%%%%%%%%%%%%%%%%%%%%%%%%%%%%%%%%%%%%%%%%%%%%%%%%%%%%%%%%%
% Einleitung
%%%%%%%%%%%%%%%%%%%%%%%%%%%%%%%%%%%%%%%%%%%%%%%%%%%%%%%%%%%%%%%%%%%%

\chapter{Introduction}
\label{Introduction}
A linear layout, in graph theory, is a way of drawing a graph where its vertices are positioned on a line (called the \textit{spine}) and its edges are arcs residing on half planes that are delimited by the spine. A linear layout is accompanied by a subdivision of the edges, where each subset occupies one half plane of the graph.\\
These two properties, the linear order of the vertices and the division of edges to half planes, characterize a linear layout. Different types of linear layouts are distinguished by the structures the edges form, the two most commonly distinguished are \textit{stack layouts} and \textit{queue layouts}.\\
One massive part of the research on linear layouts concerns \textit{stack layouts}, where each subset of edges has to form a planar graph on its respective half plane. In particular, the embedding of planar graphs in stack layouts is an important subject of research. In the past decades many great discoveries have been made in this field, the probably most notable one by Prof. Dr. Mihalis Yannakakis in 1986, who proved that all planar graphs can be embedded in a linear layout with at most four stacks.  \cite{yannakakis1986four,yannakakis1989embedding}.\\
Another important milestone regards the automatic calculation of linear layouts. In 2015 Bekos et al.\cite{Bekos2015TheBE} introduced an approach of automation that utilizes SAT-Solving. For this purpose graphs are translated into SAT instances following strict rules to define the desired linear layout. Though this results in rather large SAT instances, they can be solved in reasonable time. The discovery proved to be very useful and inspired new research.
\section{Contribution}
This thesis describes the development of a user interface, that allows the user to calculate a linear layout on the base of a graph drawing rather than a text based representation of a graph. The graph drawings are created through an easy to use editor. Additionally, to steer the SAT-Solver towards the desired linear layout, basic and advanced constraints can be imposed upon the linear layout. After passing the graph to the server application it returns a linear layout, if existent. The result is then displayed in the viewing part of the framework, which provides tools to examine the layout closely. 
\section{Thesis structure}
\autoref{PR} gives the necessary definitions that are used in this thesis and introduces linear layouts formally.
In \autoref{technologies} the technologies and tools used for the implementations are presented.
The structure and handling of the framework is explained in \autoref{Implementation}.
\autoref{Exp} explains some of the experiments for which the framework was used as a proof of concept.
Last but not least, \autoref{Discussion} recaps the evolution of the framework and draws conclusions on the outcome of the project.



